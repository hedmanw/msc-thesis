\chapter{Conclusion}
%The experiment should be complemented with a think-aloud study of merges, along with a duplicate experiment with practitioners who regularly face integration.
We presented an empirical evaluation of the novel variant integration tool \tooln, investigating the completeness of the language, and the editing efficiency and error frequency of the tool. The completeness analysis is based on variant integration commits from the open source 3D-printer software Marlin. The editing efficiency and error frequency are evaluated using a controlled experiment with 16 students. 

Our results show that there is no significant difference in terms of defects between \tooln~and two-way merging, but that much fewer editing operations are required in \tooln. However, the \tooln~tool should receive an overhaul with respect to usability, since the task completion time XXX.

More in-depth studies of variant integration should be conducted, in two directions: open-source, using developers of Marlin or projects similar to it, to elicit their XXX in a pull-based setting; and professional developers working re-engineering clone-based product lines.
With an improved user interface and with developers comfortable with intention-based integration, \tooln~should prove beneficial to the software product line re-engineering process with respect to time, editing operations, and defects.


%Gousious and coauthors have proposed tool support and machine learning approaches to support the pull-based development model.
