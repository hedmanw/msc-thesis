\chapter{Conclusion}

We presented an empirical evaluation of the novel variant integration language and tool \tooln, investigating the completeness of the language, and the beneficence of the tool, measuring editing efficiency and error frequency in variant integration. The completeness analysis is based on variant integration commits mined from the open source 3D-printer software Marlin.
The editing efficiency and error frequency are evaluated using a controlled experiment with 16 students that perform variant integration tasks using the unstructured two-way merging tool in Eclipse CDT and \tooln~on subject programs derived from \busybox~and \vim. The controlled experiment is designed to reduce learning effects and optimize for internal validity.

For completeness, all variant integration commits in Marlin can be replayed using the intentions language. Our results show that \tooln~does not perform worse than two-way merging in the number of defects introduced, but that fewer editing operations are required in \tooln. However, it should receive an overhaul with respect to usability, since the task completion time using it was much longer compared to that of Eclipse CDT.

We are optimistic that this is a stepping stone towards proper variant integration tool support that can be used in the re-engineering of software product lines (eg., virtual plaform \cite{antkiewicz2014flexible}). More in-depth studies of variant integration should be conducted, in two directions: open-source, using the developers of Marlin or projects similar to it, to elicit their insights in a pull-based setting; and professional developers working with re-engineering clone-based product lines.
With an improved user interface and with developers comfortable with intention-based integration, \tooln~should prove beneficial to the software product line re-engineering process with respect to time, editing operations, and defects.
%The results indicate a need for subsequent user studies of intention-based variant integration, and the need to further .

