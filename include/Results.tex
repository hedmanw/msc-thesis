\chapter{Results}

\section{Data collection}

In Marlin, we retrieve all 2065 merges, and extract those that were merged with conflicts, yielding 49 merges. We discard one merge that had conflicts only in non-source-code files (documentation), XX that conflicted due to whitespace changes, XX that conflicted due to configuration changes\footnote{While configuration management is indeed an important part of FOSD, in these cases, the developers have mistakenly committed their personal 3D-printer configurations into a configuration template file while committing other relevant changes, cf. Stanciulescu et al. \cite{stanciulescu2015}.}. Another XX merges were discarded because some some related artifact had syntax errors. This is summarized in Table \ref{tab:marlinmerge}.

We choose Marlin because previous work has identified important concepts in the Marlin ecosystem that we build upon, and because of access to the authors \cite{stanciulescu2015}, \cite{stanciulescu2016concepts}.

\begin{table}[h]
    \centering
    \caption{Merge commit statistics.}
    \label{tab:marlinmerge}
    \begin{tabular}{l l}
    \hline\hline
        Commit range: \texttt{99653ff..2ed1331}& \\\hline
        Nbr. of commits & 7,254\\
        $\hookrightarrow$ Nbr. of merge commits & 2,065 \\
        \hspace{1em}$\hookrightarrow$ Nbr. of conflict merge commits & 49 \\
        \hspace{2em}$\hookrightarrow$ Nbr. of useful/relevant merge commits & - \\
    \hline\hline
    \end{tabular}
\end{table}


\section{Internal evaluation}
Three developers perform 3-4 tasks, at least one of each kind.

The findings from the internal evaluation is:

\begin{itemize}
    \item The intentions suffice for performing all integration tasks. Often, just using \textit{Keep} and \textit{Remove} resolves the task.
    \item When intentions are correctly declared, then our proposed resolutions of the intentions and their actual implementations produce a correctly integrated configurable platform.
    \item Substantially fewer operations are required when using intentions, compared to the operations required when using an ordinary merge tool.
\end{itemize}

\subsection{Lessons learned}
\begin{enumerate}
    \item Provide expected output source file. Interpreting instructions is not what the experiment is about, it's about editing efficiency.
    \item Shorten files (6000 lines is too much, Vim even has 26k). 100-200 should be enough.
    \item Subjects will require training in BACONTOOL.
\end{enumerate}

\begin{table}[h]
    \centering
    \caption{Characteristics}
    \label{tab:internalchar}
    \begin{tabular}{lll|lll}
\hline\hline
\textbf{Name} & \textbf{Type} & \textbf{Files} & \textbf{$\pm$Blocks} & \textbf{$\pm$\texttt{\#ifdef}s} & \textbf{$\pm$Lines} \\
%Name & Type & Files& $\pm$ Blocks & $\pm$ \texttt{\#ifdef}s & $\pm$ Lines\\
\hline
08856d9      & Conflict     & 1 & 17    & 4     & 263   \\
17de96a      & Conflict     & 1 & 34    & 18    & 256   \\
2daa859      & Conflict     & 2 & 141 & 79      & 889   \\
3116271      & Conflict     & 1 & 6     & 0     & 51    \\
373f3ec      & Conflict     & 1 & 108 & 60      & 718   \\
46f80e8      & Conflict     & 1 & 2     & 0     & 4     \\
47c1ea7      & Conflict     & 1 & 188   & 195   & 2,343 \\
\hline
esenapaj     & Fork         & 3 & 98    & 118   & 734   \\
jcrocholl    & Fork         & 1 & 39    & 27    & 290   \\
MakerLabMe   & Fork         & 3 & 153   & 26    & 423   \\
\hline\hline
    \end{tabular}
\end{table}

\begin{table}[h]
    \centering
    \caption{Results}
    \label{tab:internalres}
    \begin{tabular}{l|llll}
\hline\hline
\textbf{Name} & \textbf{Err. Regular} & \textbf{Err. BACON?} & \textbf{\#Diff Ops.} & \textbf{\#Intentions}\\
\hline
08856d9      & 0     & 0     & 4     & 1     \\
17de96a      & 1     & 2     & 13    & 5     \\
2daa859      & 2     & 5     & 14    & 5     \\
3116271      & 1     & 1     & 15    & 5     \\
373f3ec      & 4     & 3     & 35    & 7     \\
46f80e8      & 1     & 0     & 0     & 1     \\
47c1ea7      & 1     & 1     & 5     & 4     \\
\hline
esenapaj     & 5     & 2     & 271   & 185   \\ % 23 -> 2 (column err. integration)
jcrocholl    & 0     & 2     & 116   & 44    \\
MakerLabMe   & 1     & 5     & 93    & 122   \\ % 12 -> 5  (column err. integration)
\hline\hline
    \end{tabular}
\end{table}

%Illustrated examples.

\section{\RQA}


\section{\RQB}


\section{\RQC}

