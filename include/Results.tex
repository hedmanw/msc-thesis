\chapter{A Dataset of Integration Examples}
This chapter reports on the data collection from \marlin, demonstrates replayed integrations using intentions, and reports the completeness of the intentions language.

\section{Variability-related Merges from Marlin}\label{data-coll-res}
In \marlin, we retrieve all 2065 merges, and extract those that were merged with conflicts, yielding 49 merges. We discard 2 merges that had conflicts only in non-source-code files (documentation), 2 that conflicted due to whitespace changes, 3 that conflicted due to configuration changes\footnote{While configuration management is indeed an important part of FOSD, in these cases, the developers have mistakenly committed their personal 3D-printer configurations into a configuration template file while committing other relevant changes, cf. Stanciulescu et al. \cite{stanciulescu2015}.}. Another 3 merges are discarded because some related artifact had syntax errors and could not be compiled. Additionally, 4 merges are discarded because they simply accept the mainline changes as evolution, i.e. empty changeset, and are therefore uninteresting. This is summarized in Table \ref{tab:marlinmerge}. The remaining 35 merge commits are used for subsequent analysis of integration scenarios and a subset are used for replaying in our internal tool evaluation.

\begin{table}[h]
    \centering
    \caption{\marlin~merge commit statistics}
    \label{tab:marlinmerge}
    \begin{tabular}{l r}
    \hline\hline
        Commit range: \texttt{99653ff..2ed1331}& \\\hline
        Nbr. of commits & 7,254\\
        $\hookrightarrow$ Nbr. of merge commits & 2,065 \\
        \hspace{1em}$\hookrightarrow$ Nbr. of conflict merge commits & 49 \\
        \hspace{1em}Documentation conflicts & --2\\
        \hspace{1em}Whitespace changes only & --2\\
        \hspace{1em}Configuration changes & --3\\
        \hspace{1em}Syntax errors & --3\\
        \hspace{1em}Evolution -- no merge & --4\\
        \hspace{2em}$\hookrightarrow$ Nbr. of useful/relevant merge commits & 35\\
    \hline\hline
    \end{tabular}
\end{table}

\section{Replaying Merges with Intentions}
This section demonstrates the replaying of selected chunks from \marlin. We present examples of differences from both evolution and features being reconciled. Refer to Appendix \ref{a:intentions} for the recapitulation on the intentions DSL.

Figure \ref{fig:ex-keep} shows how a change from the mainline is accepted as evolution. Figure \ref{fig:ex-remove} shows how a change from the mainline is discarded as evolution. The snippet in Figure \ref{fig:keep-as-feature} accepts a change from the fork as a feature. In Figure \ref{fig:change-pc}, two mutually exclusive changes from the mainline and fork are integrated as a feature. Last, we show the composition of intentions in Figure \ref{fig:keep-remove}.

\begin{figure}[ht]
    \centering
    \begin{minipage}{.35\textwidth}
\begin{lstlisting}[caption=Wrapped code by PC,escapechar=!]{Name}
#include "watchdog.h"
#ifndef FORK
  !\ibox{\#include "language.h"}!
#endif
#include "Sd2PinMap.h"
\end{lstlisting}
\end{minipage}\qquad
\begin{minipage}{.35\textwidth}
\begin{lstlisting}
#include "watchdog.h"
#include "language.h"
#include "Sd2PinMap.h"
\end{lstlisting}

    \end{minipage}
    \caption{\keep~intention applied (left) to replay a change from \texttt{47c1ea7 ::temperature.cpp}, ground truth and outcome (right).}
    \label{fig:ex-keep}
\end{figure}

\begin{figure}[ht]
    \centering
    \begin{minipage}{.45\textwidth}
\begin{lstlisting}[caption=Wrapped code by PC,escapechar=!]{Name}
#ifndef FORK
  !\ibox{unsigned long ms = millis();}!
#endif
if (temp_meas_ready == true) {
  [...]
}
\end{lstlisting}
\end{minipage}\qquad
\begin{minipage}{.45\textwidth}
\begin{lstlisting}
if (temp_meas_ready == true) {
  [...]
}
\end{lstlisting}

    \end{minipage}
    \caption{\remove~intention applied (left) to replay a change from \texttt{47c1ea7 ::temperature.cpp}, ground truth and outcome (right).}
    \label{fig:ex-remove}
\end{figure}


\begin{figure}[ht]
    \centering
    \begin{minipage}{.57\textwidth}
\begin{lstlisting}[caption=Wrapped code by PC,escapechar=!]{Name}
#ifdef FORK
  !\ibox{static float delta[3] = {0.0, 0.0, 0.0};}!
#endif
static float offset[3] = {0.0, 0.0, 0.0};
\end{lstlisting}
\end{minipage}\qquad
\begin{minipage}{.57\textwidth}
\begin{lstlisting}
#ifdef DELTA
  static float delta[3] = {0.0, 0.0, 0.0};
#endif
static float offset[3] = {0.0, 0.0, 0.0};
\end{lstlisting}

    \end{minipage}
    \caption{\keepasf(\texttt{DELTA})~intention applied (left) to replay a change from \texttt{373f3ec::Marlin\_main.cpp}, ground truth and outcome (right).}
    \label{fig:keep-as-feature}
\end{figure}


\begin{figure}[ht]
    \centering
    \begin{minipage}{.59\textwidth}
\begin{lstlisting}[caption=Wrapped code by PC,escapechar=!]{Name}
#ifndef !\ibox{FORK}!
  if([...]) {
    plan_buffer_line([...]);
  }
#else
  float difference[NUM_AXIS];
  for (int8_t i=0; i < NUM_AXIS; i++) {
    difference[i] = destination[i] -
      current_position[i];
  }
#endif

\end{lstlisting}
\end{minipage}\qquad
\begin{minipage}{.59\textwidth}
\begin{lstlisting}
#ifndef DELTA
  if([...]) {
    plan_buffer_line([...]);
  }
#else
  float difference[NUM_AXIS];
  for (int8_t i=0; i < NUM_AXIS; i++) {
    difference[i] = destination[i] -
      current_position[i];
  }
#endif
\end{lstlisting}

    \end{minipage}
    \caption{\changepc(\texttt{DELTA})~intention applied (left) to replay a change from \texttt{373f3ec::Marlin\_main.cpp}, ground truth and outcome (right).}
    \label{fig:change-pc}
\end{figure}


\begin{figure}[ht]
    \centering
    \begin{minipage}{.50\textwidth}
\begin{lstlisting}[caption=Wrapped code by PC,escapechar=!]{Name}
#ifndef FORK
  !\ibox{SERIAL\_ECHOLN(MSG\_PID\_AUTOTUNE\_START);}!
#else
  !\colorbox{implcolor}{SERIAL\_ECHOLN("PID Autotune start");}!
#endif
\end{lstlisting}
\end{minipage}\qquad
\begin{minipage}{.55\textwidth}
\begin{lstlisting}
SERIAL_ECHOLN(MSG_PID_AUTOTUNE_START);
\end{lstlisting}

    \end{minipage}
    \caption{\colorbox{shadecolor}{\keep}~and \colorbox{implcolor}{\remove}~intentions applied (left) to replay a change from \texttt{47c1ea7::temperature.cpp}, ground truth and outcome (right).}
    \label{fig:keep-remove}
\end{figure}

\section{RQ1: \RQA}
The intentions language can be used to replay all 35 variability-related merges in the \marlin~mainline. From these, certain chunks were listed above to illustrate the use of intentions in real scenarios. 

\rqans{\textbf{RQ1.} The proposed set of intentions indeed suffice for variant integration.}

%Additionally, the safe evolution templates for software product lines presented by Neves et al \cite{neves2011evolution} are analogous to the intentions.