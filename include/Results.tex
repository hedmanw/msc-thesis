\chapter{Results}
This chapter reports on the data collection from Marlin, the results from the internal evaluation of INCLINE, and the results of the controlled experiment. The results are tied to the the research questions.

\section{Data collection}\label{data-coll-res}
We choose Marlin as our initial source of merge examples, because previous work has identified important concepts in the Marlin ecosystem that we build upon, and because of access to the authors of \cite{stanciulescu2015}, \cite{stanciulescu2016concepts}. In Marlin, we retrieve all 2065 merges, and extract those that were merged with conflicts, yielding 49 merges. We discard 2 merges that had conflicts only in non-source-code files (documentation), 2 that conflicted due to whitespace changes, 3 that conflicted due to configuration changes\footnote{While configuration management is indeed an important part of FOSD, in these cases, the developers have mistakenly committed their personal 3D-printer configurations into a configuration template file while committing other relevant changes, cf. Stanciulescu et al. \cite{stanciulescu2015}.}. Another 3 merges are discarded because some related artifact had syntax errors and could not be compiled. Additionally, 4 merges are discarded because they simply accept the mainline changes as evolution, i.e. empty changeset, and are therefore uninteresting. This is summarized in Table \ref{tab:marlinmerge}. The remaining 35 merge commits are used for subsequent analysis of integration scenarios and a subset are used for replaying in our internal tool evaluation.

\begin{table}[h]
    \centering
    \caption{Marlin merge commit statistics}
    \label{tab:marlinmerge}
    \begin{tabular}{l r}
    \hline\hline
        Commit range: \texttt{99653ff..2ed1331}& \\\hline
        Nbr. of commits & 7,254\\
        $\hookrightarrow$ Nbr. of merge commits & 2,065 \\
        \hspace{1em}$\hookrightarrow$ Nbr. of conflict merge commits & 49 \\
        \hspace{1em}Documentation conflicts & --2\\
        \hspace{1em}Whitespace changes only & --2\\
        \hspace{1em}Configuration changes & --3\\
        \hspace{1em}Syntax errors & --3\\
        \hspace{1em}Evolution -- no merge & --4\\
        \hspace{2em}$\hookrightarrow$ Nbr. of useful/relevant merge commits & 35\\
    \hline\hline
    \end{tabular}
\end{table}


\section{Internal evaluation}
The results from the internal evaluation are shown in Table \ref{tab:internalres}. We report the number of bugs introduced by the developer in the merge in both tools, and the number of required editing operations for both tools, denoted \textit{Diff operations} and \textit{Intentions}, respectively. 

\begin{table}[h]
    \centering
    \caption{Internal evaluation results}
    \label{tab:internalres}
    \begin{tabular}{l|llll}
\hline\hline
\textbf{Name} & \textbf{Bugs Manual} & \textbf{Bugs INCLINE} & \textbf{\#Diff Ops.} & \textbf{\#Intentions}\\
\hline
08856d9      & 0     & 0     & 4     & 1     \\
17de96a      & 1     & 2     & 13    & 5     \\
2daa859      & 2     & 5     & 14    & 5     \\
3116271      & 1     & 1     & 15    & 5     \\
373f3ec      & 4     & 3     & 35    & 7     \\
46f80e8      & 1     & 0     & 0     & 1     \\
47c1ea7      & 1     & 1     & 5     & 4     \\
\hline
esenapaj     & 5     & 2     & 271   & 185   \\ % 23 -> 2 (column err. integration)
jcrocholl    & 0     & 2     & 116   & 44    \\
STM32   & 1     & 5     & 93    & 122   \\ % 12 -> 5  (column err. integration)
\hline\hline
    \end{tabular}
\end{table}

The findings from the internal evaluation is:

\begin{itemize}
    \item The intentions suffice for performing all integration tasks.
    \item When intentions are correctly declared, then our proposed resolutions of the intentions and their actual implementations produce a correctly integrated configurable platform.
    \item Substantially fewer operations are required when using intentions, compared to the operations required when using an ordinary merge tool.
\end{itemize}

Based on the experiences from this round of evaluation, we draw three conclusions that are incorporated into the experiment design for the controlled experiment: the integration goal should be provided verbatim, rather than being described in abstract terms (cf. \cite{berger2016mps}); the tasks must be smaller so that they can be performed in a reasonable time for participants; subjects will require training in \tooln.


\section{Controlled experiment}
\subsection{Editing efficiency}

\begin{figure}[ht]
    \centering
    %\includegraphics{}
    \caption{Task completion times in seconds.}
    \label{fig:completion-times}
\end{figure}

\begin{table}[ht]
    \centering
    \begin{tabular}{c|c}
         &  \\
         & 
    \end{tabular}
    \caption{Completion times, significance tests, and effect sizes}
    \label{tab:significance}
\end{table}

\begin{table}[ht]
    \centering
    \begin{tabular}{c|c}
         &  \\
         & 
    \end{tabular}
    \caption{Number of used operations.}
    \label{tab:edit-ops}
\end{table}

\begin{table}[ht]
    \centering
    \begin{tabular}{c|c}
         &  \\
         & 
    \end{tabular}
    \caption{Bugs}
    \label{tab:Bugs}
\end{table}

\begin{figure}[ht]
    \centering
    %\includegraphics{}
    \caption{Impressions of INCLINE. Intentions are intuitive, tool is mature, intention based not complex.}
    \label{fig:maturity}
\end{figure}

\subsection{Answers to open-ended q's}
Advantages  
\begin{itemize}
    \item row 4: no manual rewriting - reducing bugs and subtle differences
    \item row 7: preview reduces risk of destroying stuff
    \item row 8: easier to locate differences
    \item row 9: the tool does the work for you - lower chance of missing stuff
    \item row 11: easier to keep track of what you're doing
    \item row 12: harder to make mistakes in incline
    \item row 13: explicit concern given to variant integration
    \item row 15: preview?
\end{itemize}

Disadvantages
\begin{itemize}
    \item overall: Confusing, steep learning curve
    \item row 10: unintuitive naming
    \item row 13: offers no additional abstraction
    \item row 16: less powerful
\end{itemize}

Improvements
\begin{itemize}
    \item Useability
    \item Unintuitive naming of intentions
    \item Hotkeys!!!!
\end{itemize}

Preference question:
\begin{itemize}
    \item All respondents apart from 2 prefer INCLINE - no one prefers Eclipse - the other contenders are three-way diff
\end{itemize}

%The perceived advantages of INCLINE are that it enables a less defect-prone integration: \textit{}.

\section{\RQA}
Both \cite{neves2011evolution} and \cite{passos2016coevolution} are compatible with our intentions.

\section{\RQB}
% ANOVA cf \cite{melo2016latin} \cite{ribeiro2014emergent}

Completion times, significant difference?

There is not a significant number of differences in bugs between them. (This suggests that incline is not worse...or, hypothesis?)

Significant difference in edit ops -- or just report averages?

Question about preferred tool: Report number of people that prefer Incline. (Note also that no one prefers Eclipse.)

Attribute differences to familiarity with setting/problem + learning curve.

\section{\RQC}

