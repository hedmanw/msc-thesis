% CREATED BY DAVID FRISK, 2016
\fullt\\
\subt\\
WILHELM HEDMAN\\
Department of Computer Science and Engineering\\
Chalmers University of Technology and University of Gothenburg\setlength{\parskip}{0.5cm}

\thispagestyle{plain}			% Supress header 
\setlength{\parskip}{0pt plus 1.0pt}
\section*{Abstract}
Developing new variants by cloning is fast and simple, but as the drawbacks outweigh the early benefits, the clones must be re-engineered into an integrated platform. The challenging re-engineering process is hindered by the lack of effective tool support, because variant integration is an architectural concern, and contemporary merge tools being used for the task operate on source code as plain text. In this thesis, the recently proposed process of \textit{intention-based} variant integration is evaluated on the subject systems Marlin, BusyBox and Vim. We replay actual integration merges using the \textit{intentions} language to verify that it can be used in a real setting, followed by a controlled experiment comparing the efficiency of integration in the prototype tool \tooln~and an unstructured two-way merge tool. The results show that the intentions can capture the changes in the 35 sampled integration merges. The controlled experiment shows that for intention-based integration, fewer edit operations are required, but more actual time, and no difference can be observed for the number of defects inserted. This lays the foundation for tool improvement and subsequent user studies of intention-based integration in software product lines.


% KEYWORDS (MAXIMUM 10 WORDS)
\vfill
Keywords: happy times.

\newpage				% Create empty back of side
\thispagestyle{empty}
\mbox{}